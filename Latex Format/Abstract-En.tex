\chapter*{Abstract}

Spatial data is the data that stores geographic data types. This data is often used on systems that use data related to the territory of a region, such as the routing system or navigation system. The routing system itself has been implemented on the outdoor routing, and over the times began to be developed in the direction of the indoor routing. There are significant differences that make routing more indoor than outdoor complex routing, which is on the outdoor routing only implement routing in two dimensional spaces, while at the indoor routing allows the routing of the three dimensional spaces that represent high rise building. In the case of indoor routing, using the method of Three Dimensional Spaces will identify an object accurately by storing spatial data are represented to form undirected graph or graph is not directed to the attributes of three-dimensional data where x and y are the coordinates of a point, and z represents level height of the point. Indoor use the shortest path routing algorithm can be implemented after the three dimensional spaces structure was built in order to provide output that can be taken the shortest route between two points. This final project aims to implement routing indoor systems using the method of three dimensional spaces on School of Computing, Telkom University’s spatial data to be able to determine the shortest route in general and  close shortest route wich means the shortest route that will not pass through the open space.

\vspace{0.5 cm}
\begin{flushleft}
{\textbf{Keywords:} Spatial, Indoor Routing, Three Dimensional Spaces, Graph, Shortest Path Algorithm}
\end{flushleft}