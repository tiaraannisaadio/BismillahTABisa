\chapter*{Abstract}

Spatial data is the data that stores geographic data types. This data is often used on systems that use information related to the territory of a region, such as the routing system or navigation system. The routing system itself has been implemented on the outdoor routing, and over times it began to be developed in the direction of the indoor routing. There is significant difference that make indoor routing more complex than outdoor  routing, for example: the outdoor routing only implement routing system in two dimensional spaces, while the indoor routing allows the routing system in three dimensional spaces that represent high rise building. In the case of indoor routing, three dimensional spaces data structure will identify an object accurately by storing spatial data using undirected graph where $x$ and $y$ are the coordinates of a point on sphere, and $z$ represents height level of the point. Shortest path algorithm can be implemented then after the three dimensional spaces structure was built in order to provide output of the shortest route between two points. This final project aims to implement indoor routing system using three dimensional spaces in School of Computing, Telkom University’s spatial data to be able to determine the shortest route and  close shortest route which mean the shortest route that will not pass through the open space. 

\vspace{0.5 cm}
\begin{flushleft}
{\textbf{Keywords:} Spatial, Indoor Routing, Three Dimensional Spaces, Graph, Shortest Path Algorithm}
\end{flushleft}