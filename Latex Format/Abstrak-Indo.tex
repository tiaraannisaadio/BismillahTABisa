\chapter*{Abstrak}

Spatial database digunakan untuk penyimpanan dan pencarian data yang berhubungan dengan objek ruang antara lain titik, garis dan polgon. Spatial database memiliki beberapa jenis pencarian salah satunya adalah Nearest Neighbour. Namun, ketika Nearest Neghbour melakukan komputasi untuk pencarian membutuhkan waktu yang lama. Sehingga dapat diselesaikan menggunakan ruang partisi Voronoi Diagram. Voronoi Diagram juga memiliki kekurangan karena objek pada fragmentasi tidak dapat secara langsung ketika pencarian data. Hal ini mengakibatkan pengaksesan membutuhkan komputasi yang tinggi. Oleh karena itu, fragmentasi dibutuhkan pembagunan indexing agar dapat mengurangi untuk pencarian region. Indexing membagi partisi dengan menggunakan \textit{Highest Order Voronoi Diagram}. Pembagian partisi mengambil titik koordinat \textit{Highest Order Voronoi Diagram}. Titik tersebut dipartisi secara berulang hingga partisi tidak dapat dibagi. Kondisi tidak dapat dibagi ketika partisi hanya memiliki satu titik koordinat region. Metode ini dapat mempercepat performansi untuk menemukan region.
  
\vspace{0.5 cm}
\begin{flushleft}
{\textbf{Kata Kunci:} \textit{nearest neigbour}, \textit{region}, \textit{voronoi diagram}, \textit{spatial database}, indexing.}
\end{flushleft}